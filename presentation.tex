% lualatex presentation

\documentclass[svgnames]{beamer}
\usepackage[utf8]{inputenc}
\usepackage{fontspec}
\usepackage{arev}
\usepackage{beramono}
\usepackage{fontawesome}
\usepackage{xcolor}
\usepackage{listings}
\usepackage{hyperref}

\usetheme{default}
\setbeamertemplate{navigation symbols}
{%
%  \hspace{3em}
%  \vbox{%
%  \hbox{\insertslidenavigationsymbol}
%  \hbox{\insertframenavigationsymbol}
%  \hbox{\insertbackfindforwardnavigationsymbol}
%  \vspace{2em}}
}

\setbeamercolor{alerted text}{fg=red!70!black}
\setbeamercolor{structure}{fg=Navy}
\definecolor{wrong}{rgb}{0.7, 0, 0}
\definecolor{right}{rgb}{0, 0.5, 0}


\hypersetup{%
  pdftitle={Introduction to Version Control with Git}
  ,pdfauthor={Gert-Ludwig Ingold <gert.ingold@physik.uni-augsburg.de>}
  ,pdfsubject={Tutorial at EuroSciPy 2017, Erlangen 29.8.2017}
  ,pdfkeywords={Git, version control system, tutorial, EuroSciPy}
}

\lstset{%
  language=bash
  ,basicstyle={\ttfamily}
  ,alsoletter=$
}

\graphicspath{{./images/}}

\begin{document}

\begin{frame}

 \vspace{1truecm}
 \begin{center}
  \structure{\large\textbf{Introduction to Version Control with Git}}\\[0.3truecm]
  \structure{Gert-Ludwig Ingold}

  \vspace{1.5truecm}
  \faicon{github}\ \ttfamily{\scriptsize https://github.com/gertingold/euroscipy-git-tutorial.git}
 \end{center}
\end{frame}

\begin{frame}
 \begin{center}
  \bfseries
  \uncover<1->{Do you write 100\% bugfree code with all features implemented from the very
	       beginning?}

  \vspace{0.2truecm}
  \uncover<2->{\textbullet}

  \vspace{0.2truecm}
  \uncover<2->{Do you collaborate with others?}

  \vspace{0.2truecm}
  \uncover<3->{\textbullet}

  \vspace{0.2truecm}
  \uncover<3->{Do you want to contribute to open software?}
 \end{center}
\end{frame}

\begin{frame}{A short history of version control}
 \begin{itemize}
  \item \textit{SCCS -- Source Code Control System (1972)}
  \item \textit{RCS -- Revision Control System (1982)}\\
	single file oriented, locking mechanism
  \item \textit{CVS -- Concurrent Versions System (1990)}\\
	\textit{Subversion (2000)}\\
	centralized version control system
  \item \textit{\alert{Git}, Mercurial, Bazaar (2005)}\\
	distributed version control systems
 \end{itemize}

 \begin{center}
  \uncover<2>{\includegraphics[width=\textwidth]{git_def}}
 \end{center}
\end{frame}

\begin{frame}{Centralized version control systems}
 \begin{center}
  \includegraphics[width=\textwidth]{cvcs}
 \end{center}

 At any time, the central server contains well defined revisions
 of file sets which can be consecutively numbered.
\end{frame}

\begin{frame}{Distributed version control systems}
 \begin{center}
  \includegraphics[width=0.5\textwidth]{dvcs}
 \end{center}

 \begin{itemize}
  \item each individual repository has its own history
 \end{itemize}
\end{frame}

\begin{frame}{Distributed VCS with Gitlab / Github}
 \begin{center}
  \includegraphics[height=0.6\textheight]{dvcs-github}
 \end{center}
\end{frame}

\begin{frame}{The prime time project}
 \textit{\structure{Question}}

 How many prime numbers can be interpreted as time in the format HH:MM?

 \vspace{1\baselineskip}
 \textit{\structure{Examples}}

 \textcolor{wrong}{2179 is a prime, but 21:79 is not a valid time}\\
 \textcolor{right}{2137 is a prime and 21:37 is a valid time}\\
 \textcolor{right}{953 is a prime and 9:53 is a valid time}\\
 \textcolor{wrong}{89 is a prime, but 0:89 is not a valid time}\\
 \textcolor{right}{41 is a prime and 0:41 is a valid time}\\
 \textcolor{right}{7 is a prime and 0:07 is a valid time}

 \vspace{1.5\baselineskip}
 \begin{center}
 \uncover<2>{\structure{\bfseries And, of course, we are going to use\\ a Git repository.}}
 \end{center}
\end{frame}

\begin{frame}[fragile]{Creating a respository}
 \structure{Initializing a new repository}

 \begin{lstlisting}[backgroundcolor=\color{black!10}, language=bash]
  ~/primetime$ git init
 \end{lstlisting}

 \vspace{\baselineskip}
 \structure{What has happened?}
 \begin{lstlisting}[backgroundcolor=\color{black!10}, language=bash]
  ~/primetime$ ls -a
  .  ..  .git
  ~/primetime$ ls .git
  branches  description  hooks  objects
  config    HEAD         info   refs
 \end{lstlisting}

 \begin{center}
  \uncover<2>{\alert{Don't touch \texttt{.git} unless you know what you are doing!}}
 \end{center}
\end{frame}

\begin{frame}[fragile]{Tell Git who you are}

 \structure{Specify your name and your email address}
 \begin{lstlisting}[backgroundcolor=\color{black!10}, language=bash]
  $ git config --global user.name <your name>
  $ git config --global user.email <your email>
 \end{lstlisting}

 \vspace{\baselineskip}
 \structure{\ldots and, if you want, your preferred editor}
 \begin{lstlisting}[backgroundcolor=\color{black!10}, language=bash]
  $ git config --global core.editor <editor>
 \end{lstlisting}

 \vspace{\baselineskip}
 \structure{example configuration}
 \begin{lstlisting}[backgroundcolor=\color{black!10}, language=bash]
  $ git config --list
  user.email=gert.ingold@physik.uni-augsburg.de
  user.name=Gert-Ludwig Ingold
  core.editor=vim
  ...
 \end{lstlisting}
\end{frame}

\end{document}
